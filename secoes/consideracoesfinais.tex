\section{Considerações Finais}
   É importante entender que o termo dB pode ter diferentes significados e não tem um uma unidade fixa como as relacionadas a voltagem, metro, e afins. A unidade de dB vai depender do contexto em que ela é utilizada.

    Exemplos de diferentes intensidades de som expressas em dB(HL), ou seja, em decibéis de nível do som:
    
    \begin{itemize}
        \item 180 dB: Decolagem de foguete
        \item 140 dB: Motor à jato em movimento
        \item 120 dB: Banda de rock
        \item 110 dB: Trovoadas altas
        \item 90 dB:Tráfego urbano
        \item 80 dB: rádio no volume  bem alto
        \item 60 dB. Conversaão normal
        \item 30 dB: Susurro suave
    \end{itemize}
    
     O conceito de decibel (dB) é um muito importante e muito utilizado em Elétrica e Eletrônica para fazer a representação das mais variadas grandezas como tensão, corrente, potência, impedância, ganhos, pressão sonora, intensidade sonora, etc. Um uso muito comum de dB em eletro-eletrônica é na representação de ganhos em sistemas de amplificação, resposta de sensores, entre outros.
     