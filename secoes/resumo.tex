\begin{resumoumacoluna}
Decibel (dB) é uma unidade para expressar a razão entre duas grandezas físicas, geralmente quantidades de energia acústica ou elétrica, ou para medir a intensidade relativa dos sons. Um decibel (0.1 bel) é igual a 10 vezes o logaritmo comum da relação de potência. Então, este artigo irá informar e analisar quais são os conceitos de Bel e Decibel e suas utilidades no cotidiano. Também será falado sobre seu surgimento, como isso ajudou na sociedade, quais são suas aplicações desde então e como fazer o cálculo desta unidade de medida. Este artigo mostra como Alexander Graham Bell relacionou os sons que o ouvido escuta com uma escala de medida logaritmica, quais são seus valores absolutos e relativos e suas aplicações eletrônicas.

 \vspace{\onelineskip}
 
 \noindent
 \textbf{Palavras-chave}: bel, decibel, unidade de medida, logaritmo, som.
\end{resumoumacoluna}

\renewcommand{\resumoname}{}
\begin{resumoumacoluna}
 \begin{otherlanguage*}{english}
   \vspace{\onelineskip}
 
   \noindent
   \textbf{Keywords}: bel, decibel, unit of measurement, logarithm, sound.
   \newpage
 \end{otherlanguage*}  
\end{resumoumacoluna}